% Search for all the places that say "PUT SOMETHING HERE".

\documentclass[11pt]{article}
\usepackage{amsmath,textcomp,amssymb,geometry,graphicx,enumerate}
\usepackage{algorithm}  
\usepackage{algorithmicx}  
\usepackage{algpseudocode}
\usepackage{hyperref}
\usepackage{ctex}
\usepackage{listings}
\usepackage{xcolor}

\definecolor{codegreen}{rgb}{0,0.6,0}
\definecolor{codegray}{rgb}{0.5,0.5,0.5}
\definecolor{codepurple}{rgb}{0.58,0,0.82}
\definecolor{backcolour}{rgb}{0.95,0.95,0.92}
\lstdefinestyle{mystyle}{
	backgroundcolor=\color{backcolour},   
	commentstyle=\color{codegreen},
	keywordstyle=\color{magenta},
	numberstyle=\tiny\color{codegray},
	stringstyle=\color{codepurple},
	basicstyle=\ttfamily\footnotesize,
	breakatwhitespace=false,         
	breaklines=true,                 
	captionpos=b,                    
	keepspaces=true,                 
	numbers=left,                    
	numbersep=5pt,                  
	showspaces=false,                
	showstringspaces=false,
	showtabs=false,                  
	tabsize=2
}

\lstset{style=mystyle}

\def\Name{周盈坤}  % Your name
\def\SID{201918013229046}  % Your student ID number
\def\Class{}
\def\le{\leqslant}
\def\logN{\log{}n}
\newcommand{\ro}[1]{\romannumeral #1}
\def\Session{20春季}
\renewcommand{\algorithmicrequire}{\textbf{input:}}  
\renewcommand{\algorithmicensure}{\textbf{output:}} 
\newcommand{\norm}[1]{\lVert #1 \rVert}
\title{大数据系统与大规模数据分析19-20春季大作业Proposal}
\author{汪钇丞\ 201928015029027\ \ 刘力玮 201928015059037 \\
		汪润川\ 201928013229149\ \ 胡登杭 201928015029028 \\
	\Name\ \ \SID}
\markboth{题目8:Serverless Computing学习}{题目8:Serverless Computing学习}
\pagestyle{myheadings}
\date{}

\newenvironment{qparts}{\begin{enumerate}[{(}a{)}]}{\end{enumerate}}
\def\endproofmark{$\Box$}
\newenvironment{proof}{\par{\bf Proof}:}{\endproofmark\smallskip}
\newcommand{\angleb}[1]{\langle #1 \rangle}

\textheight=9in
\textwidth=6.5in
\topmargin=-.75in
\oddsidemargin=0.25in
\evensidemargin=0.25in

\begin{document}
\maketitle

\section{选题}
题目8:Serverless Computing学习

\section{成员}
汪钇丞\ \ 刘力玮\ \ 汪润川\ \ 胡登杭\ \ \textcolor{red}{周盈坤(组长)}

\section{组名}
Serverless Computing三巨头

\section{分工}
\begin{itemize}
	\item 周盈坤:Serverless Computing概况;Amazon AWS Lambda相关白皮书、技术文档、用户手册;Amazon AWS Lambda系统结构和工作原理,重点在于Firecracker开源项目的调研;Amazon Aurora Serverless DB介绍与概述。阅读论文: \\ \cite{hellerstein2018serverless}\underline{\textit{Serverless Computing: One Step Forward, Two Steps Back}} \\ \cite{jonas2019cloud}\underline{\textit{Cloud Programming Simplified: A Berkeley View on Serverless Computing}}
	\item 汪润川:调研Google在Serverless Computing上的架构,主要精力在于Google Cloud Run平台;阅读论文:\\ \cite{mcgrath2017serverless}\underline{\textit{Serverless Computing:
Design, Implementation, and Performance}}
	\item 汪钇丞:调研Microsoft的Serverless Computing部分,包括Azure Kubenetes Service,Azure Functions,Azure App Serverice的结构,特点、性能等。重点调研KEDA和Virtual Kubelet这两个开源项目。阅读论文:\\ \cite{wang2018peeking}\underline{\textit{Peeking Behind the Curtains
of Serverless Platforms}}
	\item 胡登杭:和汪钇丞工作类似,但主要集中在Serverless Computing容器资源的部署和auto-scaling上。主要的精力也在调研KEDA和Virtual Kubelet这两个开源项目,但和汪钇丞有内部分工。阅读论文:\\ \cite{ramakrishnan2017azure}\underline{\textit{Azure Data Lake Store:
A Hyperscale Distributed File Service for Big Data Analytics}}
	\item 刘力玮:google cloud functions与azure functions开发流程实例,基本上就是介绍两个平台上简单的HTTP请求的开发运维过程,如Github上的直接运维管理代码,以及颇为流行的CI/CD开发流水线。并针对两家公司的Serverless Computing的平台各给出并阐述一个最好的实例
\end{itemize}

\bibliographystyle{alpha}
\bibliography{doc}
\end{document}

